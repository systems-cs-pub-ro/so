\documentclass{refcard.cs.pub.ro}

\begin{document}

\raggedright
\footnotesize
\begin{multicols*}{3}

% multicol parameters <<<
% These lengths are set only within the two main columns
\setlength{\columnseprule}{0.25pt}
\setlength{\premulticols}{1pt}
\setlength{\postmulticols}{1pt}
\setlength{\multicolsep}{1pt}
\setlength{\columnsep}{2pt}
% >>>

\begin{center}
     \Large{\textbf{SO Cheat Sheet}} \\
     \Large{Lab 1 - Introducere} \\
\end{center}

\section{Linux}

\subsection{gcc}
\begin{verbatim}
gcc [-c|-S|-E] [-std=standard]
    [-g] [-pg] [-Olevel]
    [-Wwarn...] [-pedantic]
    [-Idir...] [-Ldir...]
    [-Dmacro[=defn]...] [-Umacro]
    [-o outfile] [@file] infile...
\end{verbatim}
Fazele compilării
\begin{itemize}
\item \textbf{-E}, preprocesare fişier sursă
\begin{itemize}
\item gcc -o hello.i -E hello.c
\end{itemize}
\item \textbf{-S}, compilare fişier sursă
\begin{itemize}
\item gcc -o hello.s -S hello.c
\end{itemize}
\item \textbf{-c}, asamblare fişier sursă
\begin{itemize}
\item gcc -o hello.o -c hello.c
\end{itemize}
\end{itemize}
Alte opţiuni
\begin{itemize}
\item \textbf{-Wall}, activarea tuturor avertismentelor.
\item \textbf{-lxlib}, caută biblioteca xlib la editarea de legaturi.
\item \textbf{-Ldir}, adaugă dir la lista directoarelor căutate pentru -l
\item \textbf{-Idir}, adaugă dir la începutul listei de căutare pentru fişierele antet.
\item \textbf{-DMACRO[=val]}, defineşte macro în linia de comanda.
\item \textbf{-g}, genereare simboluri de debug folosite mai târziu de debugger. 
\end{itemize}

\subsection{make}
\begin{verbatim}make [ -f makefile ] [ options ] ... [ targets ]
\end{verbatim}
\begin{verbatim}
target: dependinte
<tab>    comanda
\end{verbatim}
Variabile uzuale
\begin{itemize}
\item \textbf{CC}, defineşte compilatorul folosit
\item \textbf{CFLAGS}, defineşte flag-urile de compilare
\item \textbf{LDLIBS}, defineşte bibliotecile folosite la editarea de legături
\item \textbf{\textdollar@}, se expandează la numele target-ului
\item \textbf{\textdollar\textasciicircum}, se expandează la lista de dependinţe
\item \textbf{\textdollar\textless}, se expandează la prima dependinţă
\end{itemize}

\subsection{gdb}
\begin{verbatim}gdb prog [arguments] 
\end{verbatim}
Executabilul trebuie compilat cu flag-ul \textbf{-g}

Comenzi:
\begin{itemize}
\item \textbf{b [file:]function}, setează brakepoint la funcţia dată.
\item \textbf{r arglist}, run - rulează programul cu argumentele date.
\item \textbf{bt}, backtrace - afişează stiva programului
\item \textbf{p expr}, print - afişează valoarea lui expr 
\item \textbf{c}, continue - continuă rularea programului după oprirea la un breakpoint
\item \textbf{n}, next - execută următoare linie de program, trece peste orice apel de funcţie pe acea linie 
\item \textbf{s}, step - execută următoarea linie de program, trece prin orice apel de funcţie de pe acea linie.
\item \textbf{quit}, ieşire din GDB.
\end{itemize}
\subsection{Creare biblioteci}
\begin{itemize}
  \item \textbf{statice} 
   \begin{itemize}
     \item creare arhivă
       \begin{itemize}
         \item \textbf{ar rc libX_static.a X1.o X2.o}
       \end{itemize} 
     \item legare bibliotecă
       \begin{itemize}
         \item \textbf{gcc -Wall main.c -o main -lX_static -L.}
       \end{itemize}
   \end{itemize}
  \item \textbf{dinamice}
    \begin{itemize}
      \item \textbf{creare obiect partajat}
        \begin{itemize}
          \item \textbf{gcc -shared f1.o f2.o -o libX_shared.so}
        \end{itemize}
      \item \textbf{legare bibliotecă}
        \begin{itemize}
          \item \begin{verbatim}export LD_LIBRARY_PATH=$LD_LIBRARY_PATH:. \end{verbatim}
          \item \textbf{gcc -Wall main.c -o main -lX_shared -L.}
        \end{itemize} 
    \end{itemize}
\end{itemize}
\columnbreak
\section{Windows}
\subsection{cl}
\begin{verbatim}CL [option...] file... [option | file]... [lib...]  [/link link-opt...]\end{verbatim}
Opţiuni
\begin{itemize}
\item \textbf{/c}, realizează doar compilare fără editare de legături.
\item \textbf{/Wall}, activează toate avertismentele
\item \textbf{/D}, definire macrou în linie de comandă
\item \textbf{/I$<$dir$>$}, adaugă dir la începutul listei de căutare pentru fişierele antet.
\item \textbf{/LIBPATH$<$dir$>$}, indică editorului de legături să caute şi în dir bibliotecile pe care le va folosi programul
\end{itemize}
Specificare fişiere de ieşire
\begin{itemize}

\item \textbf{/Fa$<$file$>$}, specifică numele fişierului în limbaj de asamblare.
\item \textbf{/Fo$<$file$>$}, specifică numele fişierului obiect.
\item \textbf{/Fe$<$file$>$}, specifică numele fişierului executabil.
\end{itemize}

\subsection{nmake}
\begin{verbatim} NMAKE [option...] [macros...] [targets...] \end{verbatim}
\begin{itemize}
\item \textbf{/F}, pentru a rula alt fişier make decât cel implicit cu numele Makefile. 
\end{itemize}
\subsection{Creare biblioteci}
\begin{itemize}
  \item \textbf{statice}
    \begin{itemize}
    	 \item se foloseşte comanda \textbf{lib}
       \item \textbf{lib /out:intro.lib f1.obj f2.obj}
    \end{itemize}
  \item \textbf{dinamice}  
    \begin{itemize}
    	  \item precizarea explicită a simbolurilor folosite
    	  \begin{itemize}
    	  \item \textbf{__declspec(dllimport)}, folosit pentru a importa o funcţie
    	  \item \textbf{__declspec(dllexport)}, folosit pentru a exporta o funcţie
    	  \end{itemize}
        \item creare bibliotecă dinamică şi bibliotecă de import
        \begin{itemize}
        \item folosind opţiunea \textbf{/LD} a compilatorului cl
        \begin{itemize}
         \item \textbf{cl /LD f1.obj f2.obj}
        \end{itemize}
        \item folosind comanda link
        \begin{itemize}
        \item \textbf{link /nologo /dll /out:intro.dll /implib:intro.lib f1.obj f2.obj}
        \end{itemize}
        \end{itemize} 
    \end{itemize}  
\end{itemize}

\end{multicols*}
\end{document}

% vim: set tw=72 sw=2 sts=2 ai fdm=marker fmr=<<<,>>>:
